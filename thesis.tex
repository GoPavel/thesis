\documentclass[times,specification,annotation]{itmo-student-thesis}

\usepackage{amssymb}

\begin{document}
\studygroup{M3439}
\title{Пример оформления ВКР бакалавра}
\author{Головин Павел Андреевич}{Головин П. А.}
\supervisor{TODO}{TODO}{проф., д.т.н.}{главный научный сотрудник Университета ИТМО}
\publishyear{2020}
%% Дата выдачи задания. Можно не указывать, тогда надо будет заполнить от руки.
%\startdate{01}{сентября}{2018}
%% Срок сдачи студентом работы. Можно не указывать, тогда надо будет заполнить от руки.
%\finishdate{31}{мая}{2019}
%% Дата защиты. Можно не указывать, тогда надо будет заполнить от руки.
%\defencedate{15}{июня}{2019}

%\addconsultant{Белашенков Н.Р.}{канд. физ.-мат. наук, без звания}
%\addconsultant{Беззубик В.В.}{без степени, без звания}

\secretary{Павлова О.Н.}

%% Задание
%%% Техническое задание и исходные данные к работе
\technicalspec{TODO}

%%% Содержание выпускной квалификационной работы (перечень подлежащих разработке вопросов)
\plannedcontents{TODO}

%%% Исходные материалы и пособия
\plannedsources{\begin{enumerate}
    \item TODO
\end{enumerate}}

%%% Цель исследования
\researchaim{TODO}

%%% Задачи, решаемые в ВКР
\researchtargets{\begin{enumerate}
    \item TODO
\end{enumerate}}

%%% Использование современных пакетов компьютерных программ и технологий
%\addadvancedsoftware{Пакет \texttt{tabularx} для чуть более продвинутых таблиц}{\ref{sec:tables}, Приложения~\ref{sec:app:1}, \ref{sec:app:2}}
%\addadvancedsoftware{Пакет \texttt{biblatex} и программное средство \texttt{biber}}{Список использованных источников}

%%% Краткая характеристика полученных результатов
\researchsummary{TODO}

%%% Гранты, полученные при выполнении работы
\researchfunding{TODO}

%%% Наличие публикаций и выступлений на конференциях по теме выпускной работы
\researchpublications{TODO
%  \begin{refsection}
%    Однако покажу, как можно ссылаться на свои публикации из списка литературы:
%    \nocite{example-english, example-russian}
%    \printannobibliography
%  \end{refsection}
}

%% Эта команда генерирует титульный лист и аннотацию.
\maketitle{Бакалавр}

%% Оглавление
\tableofcontents

%% Макрос для введения. Совместим со старым стилевиком.
\startprefacepage

%:Needs:
% TODO Актуальность работы
% TODO Цели и задачи работы
% TODO Новизна работы
% TODO Практическое значение работы
% TODO Ссылки на доклады и публикации
% TODO Краткое описание структуры работы по главам

 В информатике часто используется такая математическая абстракция как бинарные отношения.
Она позволяет выразить некоторое характерное свойство пары объектов. Например, отношения порядка, отношения эквивалентности, функциональные отношения, отношение инцидентности вершин в графе и т.д.
Так же часто бинарные отношения необходимы для анализа свойства исполнения параллельных программ. Модели памяти языка программирования используют отношения порядка исполнения команд, отношение "произошло до" и другие отношения на элементарных операциях, чтобы специфицировать поведение программы при наличии гонок по данным.

Теоретически формальные модели памяти могут просто запрещать наличие гонок в коде как это было в языке Си до введения модели памяти C11 (link?). Это сильно ограничивает производительность и параллелизм программ. Поэтому сейчас языки по возможности используют в своей спецификации так называемые слабые модели памяти, которые разрешают гонки по данным и специфицируют набор возможных поведений программы. Но проблема в том, создание формальной слабой модели памяти, которая корректная применима при компиляции программы по разные архитектуры - это очень сложная задача. Так как например слабые модели C11 и JMM имеют проблему значений из воздуха (link?).

Поэтому при разработке новых модели памяти RC11 (link?) стараются использовать системы для автоматической проверки доказательств, таких как язык с зависимыми типами Coq.
Такой подход включает в себя аксиоматизация модели памяти и модели исполнения, а
также доказательство теорем о корректности компиляции на языке Coq. Это позволяет быть уверенным в правильности сложной модели памяти.

В процессе формализации модели памяти на языке Coq приходится доказывать большое количество лемм и свойств исходя из базовых аксиом. Средства Coq также позволяют автоматизировать и переиспользовать доказательства, путем обобщения доказываемых утверждений.

Так как работая с моделью памяти мы часто имеем дело с бинарными отношениями, то можно попробовать обобщить и вынести доказательства утверждений связанные с бинарными отношениями в отдельные модули и переиспользовать их. Алгебраическим обобщением бинарных отношения является Алгебра Клини (линк?).

Цель данной работы заключается в попытке упростить формальные доказательства в системе Coq связанные с моделями памяти, путем автоматизации доказательств утверждений об объектах алгебры Клини и ее расширений. Это позволит упростить разработку слабых моделей памяти в системе Coq.

%% Начало содержательной части.
\chapter{Первая глава}

  \section{Описание предметной области}

    \subsection{Алгебра отношений}
      Для любого множетсва O, можно задать алгебру отношений $ Rel<O> \subseteq P(O) $ с операциями:

      \begin{itemize}
        \item композиции ($ \cdot $) и нейтральным элементом
        \item рефлексивного тразитивного замыкания ($ \_^* $)
      \end{itemize}

      Также дополнительно мы определим операции:

      \begin{itemize}
        \item тразитивного замыкания ($ ^{+} $)
        \item пересечения($ \cap $) и объединения($ \cup $)
        \item инвертирования ($ ^\cup $) ($ x R y \Leftrightarrow y R^\cup x $)
      и пустое (0) и универсальное множество T
      \end{itemize}

    \subsection{Алгебра Клини}
      (линк на Algebra of Relation)
      Назовем алгеброй клини и будем обозначать как KA кортеж $ <A, +, \cdot, \_^*, 0, 1> $ такой, что $ <A, +, \cdot, 0, 1> $ - идемпотентное полу-кольцо и операция $ \_^* $ удовлетворяет ряду аксиом (TODO)
      Известно, что Алгебра Клини полна относительно алгебры отношений (link).

    \subsection{Алгебра Клини с тестами}
      Теперь рассмотрим очень полезное расширение KA, алгебры Клини с тестами.
      Клини алгебра с тестами (KAT), это кортеж $ <K, B, [\cdot]> $ такой, что:

      \begin{itemize}
        \item K - Клини алгебра ($ K=<A, +, \cdot, \_^*, 0, 1> $)
        \item B - булева алгебра ($ B=<B, \wedge, \vee, T, \bot> $)
        \item $ [\cdot] $ - гоморфизм из $ <A, +, \cdot, \_^*, 0, 1>  $в $ <B, \wedge, \vee, T, \bot> $
      \end{itemize}

      Чтобы сохранить полноту с бинарными отношениями расширим определение бинарных отношений, добавив к нему булеву алгебру $P(O) $ с естественной теоретико-множественной интерпретацией всех связок. А гоморфизм определим следующим образом $ [a] = {(p, p) \in a} $.
      Алгебра Клини с тестами полна относительно такой расширенной алгебры бинарных отношений. (link)
      Элементы булевой алгебры называют тестами. Они позволяют нам ограничивать наши бинарные отношения какими-то логическими выражениями.

  \section{Анализ}

  \section{Постановка задачи}

\chapter{Вторая глава}

\chapter{Третья глава}

\end{document}